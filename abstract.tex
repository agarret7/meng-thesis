%% The text of your abstract and nothing else (other than comments) goes here.
%% It will be single-spaced and the rest of the text that is supposed to go on
%% the abstract page will be generated by the abstractpage environment.  This
%% file should be \input (not \include 'd) from cover.tex.

%% The main argument of your work should appear here. Successful abstracts give
%% readers a clear idea of the project while also encouraging them to read the
%% full proposal. In your abstract you should briefly define the problem and its
%% significance, summarize your approach to solving the problem, and provide a
%% forcast of your results.

In this thesis, I designed and implemented a compiler which performs
optimizations that reduce the number of low-level floating point operations
necessary for a specific task; this involves the optimization of chains of
floating point operations as well as the implementation of a ``fixed'' point
data type that allows some floating point operations to simulated with integer
arithmetic.  The source language of the compiler is a subset of C, and the
destination language is assembly language for a micro-floating point CPU.  An
instruction-level simulator of the CPU was written to allow testing of the
code.  A series of test pieces of codes was compiled, both with and without
optimization, to determine how effective these optimizations were.
