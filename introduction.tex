%% Should contain a brief summary of the backgroud of the problem and/or overall
%% area you are investigating. You should state your particular motivation for
%% working on the present problem in your thesis as well as reasons for how and
%% why your solution(s) will contribute a new work to the field. In other words,
%% you are articulating a gap in current knowledge. Your thesis should be arguable
%% or falsifiable; the statement you make should be able to be challenged. The
%% terms that you use in your thesis should have strong, clear definitions --
%% perhaps not in the thesis statement itself, but in the same paragraph. For
%% example, your readers will want to know what you mean by "optimal",
%% "efficient", or "high performance".
%% 
%% You should be able to explain the central idea of your thesis with one
%% sentence; if that sentence is too complicated to craft, you may have too many
%% distinct ideas. Remember, your thesis is not the place where you solve all
%% problems -- focus on the original work that you are doing, which is distinct
%% form and important to other work being done in your field.

\chapter{Thesis Proposal}

\section{Introduction}

  The past decade has seen a flourishing of highly powerful computer vision
  techniques, due to the success of deep learning techniques accelerated on
  massively parallel hardware.

  Yet the shortcomings of neural techniques have only grown increasingly urgent
  in proportion to their dominance. While these approaches have resulted in the
  amortization of traditionally very difficult and high-dimensional computer
  vision problems, the lack of semantic structuring in their output has led to
  great challenges in their deployment in practical domains where
  interpretability and composability are desperately needed to integrate these
  components into larger frameworks.

  Meanwhile probabilistic techniques offer a more principled approach to
  modeling structured data, and have seen a rapid growth in the collective
  ecosystem. Such environments are desirable for their expressibility,
  corresponding to the fact that they often represent intuitive models of
  rationality inspired from cognitive and neuroscience which increasingly rely
  on such modeling to explain the human reasoning.
