\chapter{Scene Graphs}

\section{Mathematical Description}

We model the geometric state of a scene at a single time point as a \emph{scene graph}, which is a tuple $(G, \Theta, Z)$, where $G = (V, E)$ is a directed tree that encodes the scene graph \emph{structure} and $\Theta$ encodes the scene graph \emph{continuous parameters} and $Z$ encodes the scene graph \emph{discrete parameters}.
Although the framework we present can represent articulated objects, in this paper we only consider scenes involving a set of rigid objects $O$.
The scene graph structure includes a vertex $v_o \in V$ for each object $o \in O$ that represents the 6DoF pose of object $o$, as well as a vertex $r \in V$ that represents the the coordinate frame of the observer.
A directed edge $e = (v_i, v_j) \in E$ between objects $i, j \in O$ encodes that the pose of object $j$ is parametrized \emph{relative to} the pose of object $i$, by parameters $\theta_e$ and $z_e$.
The pose of object $j$ relative to object $i$ is denoted $\Delta x(z_e, \theta_e) \in SE(3)$.
A directed edge $e = (r, v_j) \in E$ from the root to an object $j \in O$ encodes that the pose of object $j$ is not parametrized relative to that of any object, but is instead a full 6DoF pose $\theta_e \in SE(3)$ where $SE(3)$ is the special Euclidean group consisting of all 6DoF poses.
The continuous and discrete parameters of the scene graph consist of the parameters for each edge in the structure ($\Theta := \{\theta_e\}_{e \in E}$ and $Z := \{z_e\}_{e \in E}$).
The set of edges $E$ must \emph{span} the set of vertices $V := \{r\} \cup \{v_o\}_{o \in O}$; that is the scene graph structure $G$ is a \emph{directed spanning tree}.

\subsection{Computing the 6DoF poses of all objects in the scene}
Given a scene graph $(G, \Theta, Z)$ the pose $x_i \in SE(3)$ of an object $i$ relative to the observer coordinate frame can be computed by walking path in the tree $G$ from the root vertex $r \in V$ to the vertex $v_i \in V$, and successively computing the pose of each object along the path from the pose of its parent.
That is, given pose $x_u \in SE(3)$ and edge $(u, v) \in E$, the pose $x_v \in SE(3)$ is computed as $x_v := x_v \cdot \Delta x(z_e, \theta_e)$ where $\Delta x(z_e, \theta_e) \in SE(3)$ is the relative pose between vertex $u$ and vertex $v$.
For an edge from the root vertex to an object vertex ($e = (r, v_i)$), $x_u := \mathbf{1}$ (the identity element of $SE(3)$) and $\Delta x(z_e, \theta_e) = \theta_e \in SE(3)$ (note that there are no discrete parameters in this case, so $z_e := ()$).
The functional form of $\Delta x(z_e, \theta_e)$ for an edge $e$ between two objects is discussed below.
One requirement is that $\Delta x(z_e, \theta_e)$ is a differentiable function of $\theta_e$; this enables inference algorithm that exploit gradient information.

\subsection{Modeling face-to-face contact between two objects}
An edge $(v_i, v_j) \in E$ from object $i$ to object $j$ indicates that the pose of object $j$ is represented relative to the pose of object $i$.
Various types of relative pose parametrizations for two objects are possible; for simplicity we model objects as polyhedra, and only model face-to-face contact between objects.
That is, an edge $e = (v_i, v_j)$ indicates that a face of object $i$ is in flush contact with a face of object $j$.
Since each object has multiple faces, the choice of which pair of faces is in contact is encoded in the discrete parameters $z_e$ for the edge.
Concretely, let $F_i$ and $F_j$ denote the faces of object $i$ and object $j$ respectively
Then, $z_e \in F_i \times F_j$.
The continuous parameters $\theta_e$ for an edge $e$ between two objects is an element of $\mathbb{R}^2 \times [0, 2 \pi)$ that contains two translational degrees of freedom ($s, t \in \mathbb{R}$, for the relative offset of the two faces) and one rotational degree of freedom ($\phi \in [0, 2 \pi)$).
For example, a cuboid object $i \in O$, the set of faces is $F_i = \left\{\mathrm{Top, Bottom, Left, Right, Front, Back}\right\}$.
For an edge $e = (i, j)$ where objects $i, j \in O$ where both objects $i$ and $j$ are cuboids, there are 36 possible values for $z_e$.

\paragraph{Slack variables for face-to-face contact}
We extend the parametrization of face-to-face contact between objects with three additional degrees of freedom of \emph{slack variables}: (i) one degree of freedom that encodes the perpendicular distance ($d \in [0, \infty)$) between the two contact faces, and (ii) two degrees of freedom for relative orientation of the two faces, encoded the surface normal unit vector $\mathbf{n}$ of the child object's face, relative to the parent object's face, which takes values on the sphere $S^2$.
Therefore, $\theta_e \in \mathbb{R}^2 \times [0, 2 \pi) \times [0, \infty) \times S^2$ for an edge $e = (v_i, v_j)$ between two objects $i$ and $j$.
Note that although this edge parametrization uses six degrees of freedom for object-to-object edges like the edge parametrization for edges from the root ($e = (r, v_j)$), the prior distribution on these parameters (described below) encourages object $j$ to be \emph{almost} in face-to-face contact with object $i$; whereas the prior distribution on the pose of object $j$ for an edge of the form $(r, v_j)$ is very different, and is typically a uniform distribution over positions within the scene bounding volume, and a uniform distribution on orientations.

\subsection{A Prior Distribution on Scene Graphs}  % alternative title; representation as a generative program
Various prior distributions on scene graphs are possible.
In our experiments, we use a generic prior distribution on scene graphs over a collection of objects $O$ that factors into two components:
(i) a prior distribution on scene graph structures $G$, denoted $p(G)$, and
(ii) a prior distribution on scene graph parameters $(Z, \Theta)$ given structure, denoted $p(Z, \Theta | G)$.
While uncertainty about the number of objects is possible to represent in our framework and implementation, for simplicity assume that the set of objects is known a-priori, and that there is one vertex for each object, and one vertex representing the observer coordinate frame, so $V$ is fixed a-priori to $\{r\} \cup \{v_o\}_{o \in O}$.
Therefore, $p(G)$ reduces to a prior distribution on edges in the scene graph.
For the prior distribution on structure, we use a uniform distribution on directed trees that are rooted at vertex $r$ and span all $|O| + 1$ vertices in the graph.
This set of directed trees is isomorphic to the set of undirected spanning trees over $|O| + 1$ vertices.
%We use a uniform distribution over forests of fixed size $N$. The structure of a forest of $N$ nodes is isomorphic to that of a spanning tree with $N+1$ nodes, where edges between the additional node and its children in the spanning tree are simply deleted.
Therefore, the prior probability of a graph $G = (V, E)$ is obtained by using Cayley's formula to count the number of undirected spanning trees on $|O| +1$ vertices:
\begin{equation}
    p(G) := p_{\mathrm{unif}(|O|)}(G) := \left\{
    \begin{array}{ll}
    (|O| + 1)^{1 - |O|} & \mbox{if $G$ is a directed spanning tree over vertices $V$ rooted at $r$}\\
    0 & \mbox{otherwise}
    \end{array}
    \right.
\end{equation}
The prior distribution on scene graph parameters factors over the edges in the scene graph:
\begin{equation}
    p(Z, \Theta | G) := \prod_{e \in E} p_e(z_e, \theta_e)
\end{equation}
where $p_e(z_e, \theta_e)$ is a probability distribution on $z_e$ and $\theta_e$ that depends on the two vertices in the edge $e = (u, v)$.
For edges $e = (r, v_j)$ where $j$ is an object, $z_e = ()$ and $p_e(z_e, \theta_e)$ is the uniform distribution on elements of $B \times SO(3)$ where $B$ is the bounding volume of the scene and $SO(3)$ is 3D rotation group (the uniform distribution on $SO(3)$ is given by the Haar measure).
For edges $e = (v_i, v_j)$ where $i$ and $j$ are objects, recall that $z_e \in F_i \times F_j$ and (using the edge parametrization including slack variables) $\theta_e = (s, t, \phi, d, \alpha_1, \alpha_2) \in \mathbb{R}^2 \times [0, 2 \pi) \times [0, \infty) \times [0, 2 \pi)^2$.
The prior distribution on edge parameters is:
\begin{equation}
    p_e(z_e, \theta_e) := \frac{1}{|F_i| |F_j|}
    \cdot p_{\mathrm{norm}(0, \sigma)}(s) \cdot p_{\mathrm{norm}(0, \sigma)}(t)
    \cdot \frac{1}{2 \pi}
    \cdot p_{\mathrm{exp}(\beta)}(d)
    \cdot \frac{1}{2 \pi}
    \cdot \frac{1}{2 \pi}
\end{equation}
(where $p_{\mathrm{norm}}$ and $p_{\mathrm{exp}}$ are the normal and exponential distribution density functions, respectively).

\section{Example Applications of Scene Graph Models}

\todo

\subsection{YCB Objects on a synthetic tabletop}

\todo

\subsection{Real YCB objects on a physical tabletop}

\todo

\subsection{Simulated objects in AI2Thor}

\todo
