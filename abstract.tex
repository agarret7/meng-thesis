% $Log: abstract.tex,v $
% Revision 1.1  93/05/14  14:56:25  starflt
% Initial revision
% 
% Revision 1.1  90/05/04  10:41:01  lwvanels
% Initial revision
% 
%
%% The text of your abstract and nothing else (other than comments) goes here.
%% It will be single-spaced and the rest of the text that is supposed to go on
%% the abstract page will be generated by the abstractpage environment.  This
%% file should be \input (not \include 'd) from cover.tex.
%% In this thesis, I designed and implemented a compiler which performs
%% optimizations that reduce the number of low-level floating point operations
%% necessary for a specific task; this involves the optimization of chains of
%% floating point operations as well as the implementation of a ``fixed'' point
%% data type that allows some floating point operations to simulated with integer
%% arithmetic.  The source language of the compiler is a subset of C, and the
%% destination language is assembly language for a micro-floating point CPU.  An
%% instruction-level simulator of the CPU was written to allow testing of the
%% code.  A series of test pieces of codes was compiled, both with and without
%% optimization, to determine how effective these optimizations were.
Recent advances in probabilistic programming have enabled the development of probabilistic generative models for visual perception using a rich abstract representation of 3D scene geometry called a scene graph.
However, there remain several challenges in the practical implementation of scene graph models, including human-editable specification, visualization, priors, structure inference, hyperparameters tuning, and benchmarking.
In this thesis, I describe the development of infrastructure to enable the development and research of scene graph models by researchers and practitioners.
A description of a preliminary scene graph model and inference program for 3D scene structure is provided, along with an implementation in the probabilistic programming language Gen.
Utilities for visualizing and understanding distributions over scene graphs are developed.
Synthetic enumerative tests of the posterior and inference algorithm are conducted, and conclusions drawn for the improvement of the proposed modeling components.
Finally, I collect and analyze real-world scene graph data, and use it to optimize model hyperparameters; the preliminary structure inference program is then tested in a structure prediction task with both the unoptimized and optimized models.
