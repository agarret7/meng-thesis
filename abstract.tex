%% The text of your abstract and nothing else (other than comments) goes here.
%% It will be single-spaced and the rest of the text that is supposed to go on
%% the abstract page will be generated by the abstractpage environment.  This
%% file should be \input (not \include 'd) from cover.tex.

%% The main argument of your work should appear here. Successful abstracts give
%% readers a clear idea of the project while also encouraging them to read the
%% full proposal. In your abstract you should briefly define the problem and its
%% significance, summarize your approach to solving the problem, and provide a
%% forcast of your results.

In this thesis, I consider Bayesian approaches to doing inference over
structured scenes. This includes the design and implementation of a
probabilistic language for representing semantic relations between multiple
objects. I propose a strategy for integrating neural techniques into MCMC-based
techniques for inverse graphics, where a generative model is constructed to
predict the behavior of neural detectors, allowing for semantic semantic
structuring of the output. I further discuss possible experimentation with a
mixed likelihood model that combines the strengths of top-down scene
structuring with analysis-by-synthesis approaches for inverse graphics. I
finally discuss the development of a framework for realtime particle filter
based inference based on the Robot Operating System and Gen probabilistic
programming language, as well as its integration into the broader Cora project
for probabilistic modeling of intuitive physics.
