%% Should contain a brief summary of the backgroud of the problem and/or overall
%% area you are investigating. You should state your particular motivation for
%% working on the present problem in your thesis as well as reasons for how and
%% why your solution(s) will contribute a new work to the field. In other words,
%% you are articulating a gap in current knowledge. Your thesis should be arguable
%% or falsifiable; the statement you make should be able to be challenged. The
%% terms that you use in your thesis should have strong, clear definitions --
%% perhaps not in the thesis statement itself, but in the same paragraph. For
%% example, your readers will want to know what you mean by "optimal",
%% "efficient", or "high performance".
%% 
%% You should be able to explain the central idea of your thesis with one
%% sentence; if that sentence is too complicated to craft, you may have too many
%% distinct ideas. Remember, your thesis is not the place where you solve all
%% problems -- focus on the original work that you are doing, which is distinct
%% form and important to other work being done in your field.

\chapter{Thesis Proposal}

\section{Introduction}

Hook: The past decade has seen a flourishing of highly powerful computer vision techniques.

Gap: Yet the shortcomings of neural techniques have only grown increasingly urgent
in proportion to their dominance. [IDENTIFY COMMON FAILURE MODES OF NEURAL
TECHNIQUES].

Motivation: Meanwhile Monte Carlo and sampling-based techniques have seen rapid
growth, with the creation of multiple probabilistic programming systems that
can leverage the power of generative modeling to produce robust and intuitive
models that are capable of expressing the problems above.
[POTENTIALLY LIST HOW BAYESIAN REASONING CAN HANDLE THE FAILURE MODES ABOVE]

In practical applications, Bayesian techniques have often been neglected due
to the computational demands of performing inference on high-dimensional latent
parameters. The lack of tools 

We propose the development of probabilistic facilities for Bayesian real-time
scene perception. In particular, we propose the development of three main components:
\begin{itemize}
  \item Toolkit corresponding to ROS pipeline, visualization, and models all
    implemented in the Gen ecosystem. [This point requires a bit of
    clarification of what exactly we're providing, as opposed to just throwing
    our development environment at the reader and saying "we made some code".]
  \item A series of problems/a dataset exploring a domain of common sense
    reasoning tasks that traditional neural techniques fail to handle.
  \item A methodology for composing generative models with bottom-up neural
    detectors in the domain of visual scene perception for fast and robust
    scene understanding.
\end{itemize}

\begin{table}
  \begin{tabularx}{\textwidth}{|X|c|}
    \hline
    \textbf{Task} & \textbf{Expected Completion} \\
    \hline
    Development of realtime ROS-based particle filter framework with visualization & Jan 15th \\
    \hline
    Two-level neuro-predictive generative model and associated inference procedure & Jan 30th \\
    \hline
    Some project involving replacing the neural component with more distributed detectors that can be modeled probabilistically (hierarchical generalization of the research from January). This needs to be broken up into more digestible tasks, and we need consideration of exactly what form this takes. & April 30th \\
    \hline
  \end{tabularx}
\end{table}
