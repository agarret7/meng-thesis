%% In this subsection, along with an outline of the work that you plan to do -- from
%% the start to the end of your project -- you should also indicate what you think
%% could possibly change as you embark on and continually work on your thesis. In
%% the outline of your work, you might want to describe your methods of data
%% collection, any hardware or software you plan to build or implement, or any
%% algorithms you design. Who you are and what you bring to your work will also
%% help define what you plan to do. Here I quote Professor Neil Spring quite
%% broadly:
%% 
%%   Provide personal insight [to your thesis proposal]. You undoubtedly have a
%%   different way of viewing the world than anyone else, perhaps more theoretical
%%   or practical or empirical or operational. Maybe you think more like a user or
%%   more like a software engineer. [Maybe you had an interesting internship or
%%   spend a summer abroad.] Perhaps your undergraduate minor shapes your
%%   worldview.
%% 
%%   Wherever this project leads you, it's what you bring to the process that
%%   makes it interesting for everyone else. Focus on techniques. Focus on the
%%   methods and how they can be applied to solve a problem. You can make an
%%   exception if conflicating or changing results motivate further analysis.
%%   Often the inputs (workload, applications, processor speeds, network speeds)
%%   will change, and so the results (performace, comparisons) and conclusions
%%   will change with them.
%% 
%% You should also indicate what kind of equipment, facilities, data, or other
%% material you may need for the completion of your work. It is imperative that
%% you provide a timeline or a clear schedule that indicates a plan for your
%% thesis work. In this plan you and your thesis supervisor should come to an
%% agreement on goals for each month of the project including (but not limited to)
%% experiments, data collection, analysis, any refining, drafting of thesis, final
%% results, and revision of thesis. You are welcome to insert a chart with a
%% summary of your goals for each month. Most EECS Master's degree theses are
%% assigned a total number of 360 hours. We ask that you plan accordingly.

\section{Proposed Work}

\begin{table}
  \begin{tabularx}{\textwidth}{|X|c|}
    \hline
    \textbf{Task} & \textbf{Expected Completion} \\
    \hline
    Development of realtime ROS-based particle filter framework with visualization & Jan 15th \\
    \hline
    Two-level neuro-predictive generative model and associated inference procedure & Jan 30th \\
    \hline
    Some project involving replacing the neural component with more distributed detectors that can be modeled probabilistically (hierarchical generalization of the research from January). This needs to be broken up into more digestible tasks, and we need consideration of exactly what form this takes. & April 30th \\
    \hline
  \end{tabularx}
\end{table}

\todo[FIG: Bayes' net diagram 
          SUBFIG A: static structure, dynamic parameters
          SUBFIG B: dynamic structure, dynamic parameters]

\subsection{Composing Neural Techniques with Generative Modeling}

\todo

\subsection{Fixed Scene Structure, Dynamic Parameters}

\todo

\subsection{Dynamic Scene Structure, Dynamic Parameters}

\todo

\subsubsection{Reversible Jump MCMC}

\todo

This part is a highly relevent research goal, due to the undervaluing of
semantic structuring of information by the neural network community. In some
sense we are trying to provide a solution to a problem that is not fully
visible to the community, or at the very least is thought to be "too hard" for
current techniques. Hard grounding metrics, or at least a fuller range of
explicitly specified common-sense tasks will go a long way in changing the
over-skeptical perspective toward Bayesian techniques trying to answer this
question. Thus, an important part of the thesis work will be identifying the
gap that exists between types of problems that we'd like to solve (common sense
reasoning), and how neural networks currently fail to address these problems,
as well as providing as much of a solution as we can in the time available,
which we believe will be compelling enough to demonstrate the utility of
Bayesian methods in modern CV pipelines.
